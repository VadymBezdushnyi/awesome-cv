%-------------------------------------------------------------------------------
%	SECTION TITLE
%-------------------------------------------------------------------------------
\cvsection{Projects}

%-------------------------------------------------------------------------------
%	CONTENT
%-------------------------------------------------------------------------------
\begin{cventries}

% %---------------------------------------------------------
%  \cventry
%   {New Linux kernel} % Short description
%   {Ninux} % Project title
%   {1 hour} % Amount of work
%   {1k engineers} % Amount of engineers
%   {
%     \begin{cvitems} % Description(s) of tasks/responsibilities
%       \item {Implemented Batching Service to compile batches out of targets.}
%       \item {Performed research batching strategies of Batching Service.}
%       \item {Implemented different approaches based on memory/time estimates.}
%       \item {Developed model to estimate peak heap memory usage of build.}
%     \end{cvitems}
%     % \begin{cvsubentries}
%     %  \cvsubentry{}{KNOX(Solution for Enterprise Mobile Security) Penetration Testing}{Sep. 2013}{}
%     %  \cvsubentry{}{Smart TV Penetration Testing}{Mar. 2011 - Oct. 2011}{}
%     % \end{cvsubentries}
%   }

%---------------------------------------------------------
  \cventry
    {based on sequential method}
    {Resolution Theorem Proving}
    {2 month}
    {2 engineers}
    {
      \begin{cvitems}
        \item {Service to make resolution proofs using classic first-order logic. Generates counterexample if one exists.}
        \item {Java; Javascript; Spring Boot; D3.js; MathJax; RESTfull}
      \end{cvitems}
    }
  \cventry
    {neural network evolution}
    {Smart Pacmans}
    {2 month}
    {1 engineer}
    {
      \begin{cvitems}
        \item {Alternative method to train neural network using genetic algorithms.}
        \item {Watch how neural networks evolve in the environment with basic laws of nature(selection, mutation, crossing).}
        \item {Based on own low-level library for neural network manipulation.}
        \item {C\#; C++; XNA Game Studio}
      \end{cvitems}
    }
%---------------------------------------------------------
\end{cventries}
